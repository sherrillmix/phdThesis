\documentclass[../sherrill-Mix_thesis.tex]{subfiles}
\begin{document}
\graphicspath{{im/}{latency/im/}}

\newcommand{\Active}{Active \cdFour} %\active already taken
\newcommand{\Resting}{Resting \cdFour}
\newcommand{\Memory}{Central Memory \cdFour}
\newcommand{\Bcl}{Bcl-2 transduced \cdFour}
\newcommand{\Jurkat}{Jurkat}
\newcommand{\nFeatures}{140}
\newcommand{\nSites}{12,436}


\chapter{HIV latency and integration site placement in five cell-based models}
\label{chapLatency}
 
\textBox{This chapter was originally published as:\\
	\citeBox{Sherrill-Mix2013}
	I led the computational analysis, with assistance from CC Berry and N Malani. MK Lewinski, D Looney and J Guatelli analyzed integration sites using IonTorrent sequencing. M Famiglietti, A Bosque and V Planelles prepared DNA from latent and activated T cells using the Central Memory CD4 + model. L Shan, RF Siliciano, MJ Pace, LM Agosto, KE Ocwieja and U O'Doherty contributed data and suggestions. FD Bushman and I planned the overall study. I produced the figures. FD Bushman and I wrote the paper.

	Additional files are available at \url{http://www.retrovirology.com/content/10/1/90/additional}
}

\section{Abstract}
	Background: HIV infection can be treated effectively with antiretroviral agents, but the persistence of a latent reservoir of integrated proviruses prevents eradication of HIV from infected individuals. The chromosomal environment of integrated proviruses has been proposed to influence HIV latency, but the determinants of transcriptional repression have not been fully clarified, and it is unclear whether the same molecular mechanisms drive latency in different cell culture models. 

	Results: Here we compare data from five different \textit{in vitro} models of latency based on primary human T cells or a T cell line.  Cells were infected in vitro and separated into fractions containing proviruses that were either expressed or silent/inducible, and integration site populations sequenced from each. We compared the locations of 6,252 expressed proviruses to those of 6,184 silent/inducible proviruses with respect to 140 forms of genomic annotation, many analyzed over chromosomal intervals of multiple lengths. A regularized logistic regression model linking proviral expression status to genomic features revealed no predictors of latency that performed better than chance, though several genomic features were significantly associated with proviral expression in individual models.  Proviruses in the same chromosomal region did tend to share the same expressed or silent/inducible status if they were from the same cell culture model, but not if they were from different models. 
	
	Conclusions: The silent/inducible phenotype appears to be associated with chromosomal position, but the molecular basis is not fully clarified and may differ among \textit{in vitro} models of latency.

\section{Background}
	Highly active antiretroviral therapy (HAART) can suppress HIV-1 replication in infected patients, but the ability of HIV to persist as an inducible reservoir of latent proviruses \citep{Chun1995, Chun1997, Davey1999} obstructs eradication of the virus and functional cure \citep{Richman2009}. These latent proviruses are long lived \citep{Finzi1999,Siliciano2003} and relatively invisible to the immune system \citep{Finzi1997,Chun1997}. The potential for even a single virus to restart infection despite successful antiviral therapy means that it may be necessary to eliminate all latent proviruses to eradicate HIV from an infected person. %  Achieving eradication requires a better understanding of the causes of latency.


	After integration, a positive feedback loop of Tat transactivation appears to partition proviral gene activity into either of two stable states \citep{Weinberger2008,Singh2010,Razooky2011}---abundant Tat driving high proviral expression or little Tat leading to quiescent latency. Similar to the positional effect variegation observed in fruit fly chromosomal rearrangements \citep{Muller1930,Gaszner2006}, studies on cell clones with single integrations show that differing integration sites can have large differences in proviral expression \citep{Jordan2001,Jordan2003,Pearson2008}. These data suggest that integration site location, along with the cellular environment \citep{Romerio1997,Coull2000,He2002,Pearson2008}, influences the balance between latency and proviral expression.  % along with the cellular environment, including T cell activation status, and epigenetic modifications of the chromatin of the provirus \citep{Romerio1997,Coull2000,He2002,Pearson2008}, 
	%that are preserved even after selecting clones with particularly high or low expression and allowing them to regrow 


	 Associations between latency and genomic features have also been reported in collections of integration sites from cell culture models although the consistency of these effects across model systems and their relationships to latency in patients remains uncertain. \citet{Lewinski2005} reported that proviruses integrated in gene deserts, alphoid repeats and highly expressed genes are more likely to have low expression. \citet{Shan2011} reported an association between latency and integration in the same transcriptional orientation as host genes. \citet{Pace2012} found that silent and expressed provirus integration sites differed in the abundance and expression levels of nearby genes, GC content, CpG islands and alphoid repeats.  In model systems with defined integration sites, \citet{Lenasi2008} reported decreased and \citet{Han2008} reported increased viral transcription when the provirus is downstream of a highly expressed host gene. 

	Cell-based models of latency are important for many aspects of HIV research, including screening small molecules that can reverse latency and potentially allow eradication \citep{Shan2012,Boehm2013}. Location-driven differences in expression are preserved even after demethylation and histone deacetylase treatment \citep{Jordan2001}, which suggests that integration location has the potential to confound ``shock and kill'' anti-latency treatments \citep{Savarino2009,Archin2012}. A greater understanding of the effects of integration site location on latency could thus affect antiretroviral development. %[Archin2012 didn't say shock and kill but Deeks2012 review of Archin2012 do. May have been coined by Hamer2004 but paper not available]]

	%This study provides the first cross study comparison of the effects of integration site and latency.
	To search for features of integration site associated with latency, we generated a set of inducible and expressed integration sites using a primary central memory \cdFour{} T cell model of latency \citep{Bosque2009,Bosque2011}, collected four previously reported integration site datasets and modeled the effects of genomic features near the integration site on the expression status of these proviruses.  Although some genomic features associated with latency in individual models, no feature was consistently associated with proviral expression across all five cell culture models. However, closely neighboring proviruses within the same cellular model shared the same latency status much more often than expected by chance suggesting that chromosomal position of integration affects latency but that the mechanism remains unclear or differs between cell culture models. Thus these data help inform the design of experiments in HIV eradication research.

\section{Methods}
	\subsection{Integration sites}
		Naive \cdFour{} T cells were purified by negative selection from peripheral blood mononuclear cells. The cells were activated with anti-CD3 and anti-CD28 (+TGF-beta, anti-IL-12, and anti-IL-4) to generate ``non-polarized'' cells (the in vitro equivalent of central memory T cells). Five days after isolation, cells were infected with an NL4-3-based virus with GFP in place of Nef and the LAI envelope (X4) provided in trans at a concentration of 500 ng of p24 as measured by ELISA per million cells. Based on previous experience with this model, this amount of p24 should produce an MOI of approximately 0.15. Cells were cultured in the presence of IL-2. Two days post-infection, cells were sorted for GFP+; this active population expresses GFP even when treated with flavopiridol, although for this study they were not treated. The inducible population was the set of GFP negative cells from the initial sort that, 9 days post-infection, were activated with anti-CD3 and anti-CD28 and sorted for GFP production. 

		Genomic DNA from the inducible and expressed populations was digested with MseI, ligated to an adapter, and amplified by ligation-mediated PCR essentially as in \citet{Wu2003} and \citet{Mitchell2004}  except that the nested PCR primers included sequence for the Ion Torrent P1 adapter and adapter A sequence with a 5 base barcode sequence specific to the inducible or expressed conditions. Amplicons were sequenced using an Ion Torrent Personal Genome Machine (PGM) according to manufacturer's instructions using an Ion 316 chip and the Ion PGM 200 Sequencing kit (Life Technologies). The sequence reads were sorted into samples by barcode. All reads were required to match the expected $5'$ sequence with a Levenshtein edit distance less than 3 from the expected barcode, $5'$ primer and HIV long terminal repeat (LTR). The $5'$ primer and HIV sequence, along with the $3'$ primer if present, were trimmed from the read. Sequences with less than 24 bases remaining or containing any eight base window with an average quality less than 15 were discarded. Duplicate reads and reads forming an exact substring of a longer read were removed. 

	\subsection{Analysis}
		All statistical analysis was performed in R 2.15.2 \citep{RCoreTeam2012}. The analyses are described in a reproducible report (Appendix \ref{appLatencyReport}). The annotated integration site data necessary to perform the analyses and the compilable code to generate this reproducible report are provided as supplemental information \citep{Sherrill-Mix2013}. The new \Memory{} data set was analyzed as in \citet{Berry2006}.  The integration patterns appeared similar to previously reported HIV integration site datasets \citep{Wang2007a}.

	\subsection{Previously published data}
		We collected integration sites from three previously reported studies (Table \ref{tabLatencySamples}), for a total of four expressed versus silent/inducible pairs of samples. These studies used primary \cdFour{} T cells or Jurkat cells infected with HIV or HIV-derived constructs as cell culture models of latency. Flow cytometry allowed cells expressing viral encoded proteins to be sorted from non-expressing cells. In two of the studies, these non-expressing populations were stimulated to ensure that the provirus could be aroused from latency. Specific differences in protocol between the study sets are summarized below. 

	\begin{description}
		\item[\Jurkat{}] \citet{Lewinski2005} infected Jurkat cells with a VSV-G pseudotyped, GFP-expressing pEV731 HIV construct (LTR-Tat-IRES-GFP) \citep{Jordan2001} at an MOI of 0.1. The cells were sorted into GFP+ and GFP- two to four days after infection. GFP+ cells were sorted again two weeks after infection and cells that were again GFP+ were collected for integration site sequencing. GFP- cells were sorted for GFP negativity twice more then stimulated with TNF$\alpha$. Cells that were GFP+ after stimulation were collected for integration site sequencing. DNA was digested with MseI or a combination of NheI, SpeI and XbaI, ligated to adapters for nested PCR, amplified and sequenced by Sanger capillary electrophoresis. 
		\item[\Bcl{}] \citet{Shan2011} transduced \cdFour{} T cells with Bcl-2, costimulated with bound anti-CD3 and soluble anti-CD28 antibodies, interleukin-2 and T cell growth factor and then infected with X4-pseudotyped GFP-expressing NL4-3-$\delta$6-drEGFP construct \citep{Mochizuki1998} at an MOI of less than 0.1. DNA was extracted, digested with PstI and circularized \citep{Han2004}. HIV-human junctions were amplified by reverse PCR and sequenced using Sanger capillary electrophoresis.
		\item[\Active{} \& \Resting{}] \citet{Pace2012} spinoculated \cdFour{} T cells with HIV NL4-3 at an MOI of 0.1. After 96 hours, the cells were stained for intracellular Gag CD25, CD69 and HLA-DR and sorted into four subpopulations based on activation state and Gag expression; activated Gag-, activated Gag+, resting Gag- and resting Gag+. The ability of the viruses to reactivate was not tested although previous studies have shown that the majority are likely inducible \citep{Plesa2007}. Genomic DNA was extracted and digested with restriction enzymes MseI and Tsp509 and ligated to adapters. Proviral LTR-host genome junctions were sequenced by 454 pyrosequencing after nested PCR.
	\end{description}

		All datasets were processed using the hiReadsProcessor R package \citep{Malani}. Adaptor trimmed reads were aligned to UCSC freeze hg19 using BLAT \citep{Kent2002}. Genomic alignments were scored and required to start within the first three bases of a read with 98\% identity. Alignments for a given read with a BLAT score less than the maximum score for that read were discarded.  Reads giving rise to multiple best scoring genomic alignments were excluded, while reads with a single best hit were dereplicated and converged if within 5bp of each other. The \Bcl{} sample was sequenced from U3 in the $5'$ HIV LTR while the other samples were sequenced from U5 in the $3'$ LTR. To account for the 5 base duplication of host DNA caused by HIV integration, the chromosomal coordinates of the \Bcl{} sample were adjusted by $\pm 4$ bases. %In all analyses except the alphoid repeat section, only integration sites that mapped to a single unique location in the genome were used. 
		
		To allow for alignment difficulties in the analysis of genomic repeats, reads with multiple best scoring alignments, along with the single best hit reads used above, were included in the repeat analyses. If any best scoring alignment for a read fell within a repeat, then that read was considered to map to that repeat. 
		
		% To attempt to reduce these repetitive reads to likely integrations, all reads with any alignment to the same chromosomal location [+-X NIRAV]] were assumed to originate from a single integration. [DETAILS ON ALGORITHM]] %The algorithm for assigning an ISU ID is as follows: for each read check how many best genomic locations are present. Separate reads with only location from the rest. If any read with $>1$ location match to any read with only one location, then tag both reads with the same ID. For rest of the untagged reads, if any location from a read with $>1$ location matches to any other reads with $>1$ location, then tag the same ID to all reads under that collection. Repeat the last step until all reads have been tagged. [WORK HERE]]

	\begin{table}
	\resizebox{\columnwidth}{!}{
	\small
		\begin{tabular}{|p{0.49in}|p{0.53in}|p{0.8in}|P{0.6in}|p{0.6in}|P{0.75in}|P{0.59in}|p{0.59in}|p{0.6in}|}
			\hline
				\textbf{\textbf{Title}    } & \textbf{Cell type}                         & \textbf{Virus}                                                                           & \textbf{Time of harvest after infection} & \textbf{Se\-quenc\-ing} & \textbf{Gener\-ation of expressed vs. silent/ inducible} & \textbf{Cita\-tion}    &  \textbf{Silent/ in\-duc\-ible unique sites} & \textbf{Ex\-press\-ed unique sites} \\ 
			\hline     
			\Jurkat{}             & Jurkat cells                               & HIV vector pEV731 (LTR-Tat-IRES-GFP)                                                     & 2 weeks                                  & Sanger                     & TNF$\alpha$, GFP expression                            & \citet{Lewinski2005} &  463\newline inducible                                    & 643                                 \\ 
			\hline     
			\Bcl{} & Primary \cdFour{} T cells (Bcl-2 trans\-duced) & HIV NL4-3-$\delta6$-drEGFP (inactivated \textit{gag}, \textit{vif}, \textit{vpr}, \textit{vpu}, \textit{nef} and \textit{env} replaced by GFP) & 3 days + 3-4 weeks + 3 days              & Sanger                     & anti-CD3, anti-CD28 antibodies, GFP expression         & \citet{Shan2011}     &  446\newline inducible                                    & 273                                 \\ 
			\hline     
			\Active{}         & Primary active \cdFour{} T cells             & HIV NL4-3                                                                                & 3 days                                   & 454                        & high vs. low Gag                                       & \citet{Pace2012}     &  1604\newline silent                                    & 1274                                \\ 
			\hline     
			\Resting{}        & Primary resting \cdFour{} T cells            & HIV NL4-3                                                                                & 3 days                                   & 454                        & high vs. low Gag                                       & \citet{Pace2012}     &  1942\newline  silent                                   & 784                                 \\ 
			\hline     
			\Memory{}         & Primary central memory \cdFour{} T cells             & HIV NL4-3 $\Delta$Nef GFP                                                                      & 2 days/9 days                            & Ion\-Torrent                 & anti-CD3, anti-CD28 antibodies, GFP expression         & This paper           &  1729\newline inducible                                   & 3278                                \\ 
			\hline
		\end{tabular}
	}
	\caption[Integrations from \textit{in vitro} models of latency]{HIV-1 integration datasets from \textit{in vitro} models of latency where the proviruses were determined to be silent/inducible or expressed}
	\label{tabLatencySamples}
	\end{table}

	\subsection{Genomic features}
		A total of \nFeatures{} whole genome features for \cdFour{} T-cells were gathered from data sources indicated in Table \ref{tabVariables}.  For features encoded as peaks or hotspots, the log of the distance of each integration site to the nearest border was used for modeling. Integration sites from HIV 89.6 infection in primary \cdFour{} T cells \citep{Berry2014} were used to count nearby integrations and determine a $\pm20$bp position weight matrix for integration targets. Illumina RNA-Seq from active \cdFour{} cells (Chapter \ref{chapRnaSeq}) was used to estimate raw cellular expression and fragments per kilobase of transcript per million mapped reads for genes as calculated by Cufflinks \citep{Trapnell2010}. For sequence-based data like RNA-Seq and ChIP-Seq, the number of reads aligned within a $\pm$ 50, 500, 5,000 50,000 and 500,000 bp windows of each integration site were counted and log transformed. In addition, chromatin state classifications derived from a hidden Markov model based on histone marks and a few binding factors \citep{Ernst2010} were included as binary variables. All data from previous genomic freezes were converted to hg19 using liftover \citep{Hinrichs2006}. 

	\begin{table}
		\resizebox{\columnwidth}{!}{
		\small
		\begin{tabular}{|p{1.2in}|l|p{0.79in}|l|P{2.3in}|}
			\hline
			\textbf{Group}    & \textbf{Type}  & \textbf{Source}               & \textbf{Number} & \textbf{Types}                                                                                                                                                                                                   \\ 
			\hline
			T cell expression & RNA-Seq        & Chapter \ref{chapRnaSeq}      & 1               & RNA                                                                                                                                                                                                              \\ 
			\hline
			Jurkat expression & RNA-Seq        & Encode \citep{Rosenbloom2013} & 1               & wgEncodeHudsonalphaRnaSeq                                                                                                                                                                                        \\ 
			\hline
			Integration sites & Locations      & \citet{Berry2014}             & 1               & sites                                                                                                                                                                                                            \\ 
			\hline
			DNase sensitivity & DNA-Seq/peaks  & Encode \citep{Rosenbloom2013} & 1               & wgEncodeOpenChromDnase                                                                                                                                                                                           \\ 
			\hline
			Methylation       & DNA-Seq        & \citep{Han2012}               & 1               & Methyl                                                                                                                                                                                                           \\ 
			\hline
			CpG               & Locations      & UCSC \citep{Meyer2013}        & 1               & cpgIslandExt                                                                                                                                                                                                     \\ 
			\hline
			Sequence-based    & Continuous     & ---                           & 4               & \% GC, HIV PWM score, distance to centrosome, chromosomal position                                                                                                                                               \\ 
			\hline
			Repeats           & Locations      & UCSC \citep{Meyer2013}        & 16              & DNA, LINE, Low\_complexity, LTR, Other, RC, RNA, rRNA, Satellite, scRNA, Simple\_repeat, SINE, snRNA, srpRNA, tRNA, alphoid                                                                                      \\ 
			\hline
			Histone features  & ChIP-Seq/Peaks & \citet{Wang2008a}             & 18              & H2AK5ac, H2AK9ac, H2BK120ac, H2BK12ac, H2BK20ac, H2BK5ac, H3K14ac, H3K18ac, H3K23ac, H3K27ac, H3K36ac, H3K4ac, H3K9ac, H4K12ac, H4K16ac, H4K5ac, H4K8ac, H4K91ac                                                 \\ 
			\hline
			Histone features  & ChIP-Seq/Peaks & \citet{Barski2007}            & 23              & CTCF, H2AZ, H2BK5me1, H3K27me1, H3K27me2, H3K27me3, H3K36me1, H3K36me3, H3K4me1, H3K4me2, H3K4me3, H3K79me1, H3K79me2, H3K79me3, H3K9me1, H3K9me2, H3K9me3, H3R2me1, H3R2me2, H4K20me1, H4K20me3, H4R3me2, PolII \\ 
			\hline
			Chromatin state   & Binary         & \citet{Ernst2010}             & 51              & state$_1$,state$_2$,...,state$_{51}$                                                                                                                                                                             \\ 
			\hline
			HATs and HDACs    & ChIP-Seq       & \citet{Wang2009a}             & 11              & Resting-HDAC1, Resting-HDAC2, Resting-HDAC3, Resting-HDAC6, Resting-p300, Resting-CBP, Resting-MOF, Resting-PCAF, Resting-Tip60, Active-HDAC6, Active-Tip60                                                      \\ 
			\hline
			Nucleosome        & ChIP-Seq       & \citet{Schones2008}           & 2               & Resting-Nucleosomes, Active Nucleosomes                                                                                                                                                                          \\ 
			\hline
			UCSC genes        & Locations      & \citet{Hsu2006}               & 4               & in gene, in gene (same strand), gene count, distance to nearest gene, in exon, in intron                                                                                                                         \\ 
			\hline
		\end{tabular}
		}
		\caption[Genomic data available for comaprison to integration sites]{Genomic data available for comparison to HIV integration sites}
		\label{tabVariables} \end{table} %\hline 


\section{Results}
	The combination of integration site data newly reported here (set named ``\Memory{}'') with previously published data (sets named ``\Jurkat{}'', ``\Bcl{}'', ``\Active{}'', and ``\Resting{}'') provides a collection of \nSites{} integration sites (Table \ref{tabLatencySamples}) where the expression status of the provirus---silent/inducible or expressed---is known. In three of the datasets, \Jurkat{}, \Memory{} and \Bcl{}, the proviruses were sorted based on inducibility. In the \Resting{} and \Active{} datasets, cells were sorted only based on proviral expression. Previous studies have shown that most silent proviruses in this model system are inducible \citep{Plesa2007}.

	\subsection{Global model}
		If a genomic feature and latency are monotonically related then we should be able to detect this relationship using Spearman rank correlation. In addition if a feature has a consistent effect across models we should see a consistent pattern in the direction of correlation. A simple first look for correlation between genomic features (Table \ref{tabVariables}) and latency status yielded inconsistent results among the five samples with no variables having a significant Spearman rank correlation across all, or even four out of five, of the samples (Figure \ref{figCor}). This suggests that there is not a consistent simple monotonic relationship between the genomic variable and latency, or that any such correlations are modest and not detectable across all studies given the available statistical power. We return to some of the stronger trends below.

	%correlation
	\begin{figure}
		\centering
			\includegraphics[width=\textwidth]{variableCorrelationDerep.pdf} %REMOVE%
		\caption[Correlations of genomic features and latency]{Spearman rank correlation between proviral expression status and genomic features. Only genomic features with at least one correlation with latency with a false discovery rate $q$-value $<0.01$ (marked by asterisks) are shown.}
		\label{figCor}
	\end{figure}

		To investigate whether a combination of variables may affect latency, we fit a lasso-regularized logistic regression, as implemented in the R package glmnet \citep{Friedman2010}, to predict latency using the genomic variables. The relationship between silent/inducible status and each genomic variable was allowed to vary between models by including the interaction of genomic features with dummy variables indicating cellular model. The $\lambda$ smoothing parameter of the lasso regression was optimized by finding the $\lambda$ with lowest classification error in 480-fold cross validation and finding the simplest model with misclassification error within one standard error.

		The proportion of silent/inducible sites varied between the samples. To avoid the model overfitting on this source of variation, an indicator variable for each sample was included in the base model. The base model with no genomic variables was selected as the best model by cross validation (Figure \ref{figIndivFit}A). This suggest that there is not a consistent linear relationship between an additive combination of genomic variables and latency across all models.

		\begin{figure}
			\centering
				\includegraphics[width=1\textwidth]{individualFits.pdf} %REMOVE%
			\caption[Lasso regressions predicting latency]{Misclassification error from cross validation for lasso regressions of silent/inducible status on genomic features as a function of $\lambda$, the regularization coefficient for the lasso regression, for all cell culture models combined and each individual cell culture model. The number of variables included and size of coefficients in the model increases to the left. Whiskers show the standard error of mean misclassification error. Dashed vertical lines indicate the minimum misclassification error and the simplest model within one standard error. Dotted horizontal line indicates the misclassification error expected from random guessing.}
			\label{figIndivFit}
		\end{figure}


		When each dataset was fit individually with leave-one-out cross validation, improvements in cross-validated misclassification error were only observed in the \Active{} (5.8\% decrease in misclassification error, standard error: 2.1) and \Jurkat{} (6.7\% decrease in misclassification error, standard error: 3.5) samples (Figure \ref{figIndivFit}B-F). There was no overlap in variables selected for the \Active{} and \Jurkat{} samples. %(Additional File 1).

		Finding little global association between latency and genomic features, we investigated whether predictors of latency reported previously by single studies were consistently associated with latency across studies.

	\subsection{Cellular transcription}
		Model systems with defined integration sites show upstream transcription can interfere with viral transcription \citep{Greger1998} and that cellular transcription in the same orientation may interfere with viral transcription \citep{Lenasi2008} or increase viral transcription \citep{Han2008} and in opposite orientations may decrease transcription \citep{Han2008}. In integration site studies, integration outside genes appears to increase latency \citep{Lewinski2005} but high transcription of nearby host cell genes may cause increased latency \citep{Lewinski2005,Shan2011}. In addition, Tat or other viral proteins may affect cellular transcription \citep{Marco2008,Chang2011}. %%NEW

		To look at transcription and latency, we ran a logistic regression of silent/inducible status on a quartic function of RNA expression, as determined by RNA-Seq reads within 5,000 bases in Jurkat cells for the \Jurkat{} sample or \cdFour{} T cells for the remaining samples, interacted with indicator variables encoding cell culture model. There appears to be little agreement between samples (Figure \ref{figRnaGlm}). The \Resting{} and \Active{} datasets show an enrichment in silent proviruses in regions with low gene expression. The other three studies show the opposite or no relationship for low expression regions. The two samples showing increased silence in areas of low expression  (\Resting{} and \Active{}) are from a study that did not check whether inactive viruses could be activated. One possible explanation is that regions with low gene transcription may harbor proviruses that are not easily activated, though some other discrepancy between \textit{in vitro} systems could also explain the difference. Both the \Jurkat{} and \Active{} samples appear to increase in latency with increasing expression while the remaining three studies did not show a strong trend. 

	\begin{figure}
		\centering
			\floatbox[{\capbeside\thisfloatsetup{capbesideposition={right,top}}}]{figure}[\FBwidth]{
				\includegraphics[width=0.5\textwidth]{rnaGlm.pdf} %REMOVE%
			}{
				\caption[Cellular expression and latency]{Predictions from a logistic regression of silent/inducible status on cellular RNA expression. High y-axis values are predicted to be silent/inducible. Dashed line shows where equal odds of silent/inducible and expressed are predicted. Solid lines show predictions from the regression for each sample and shaded regions indicate one standard error from the modeled predictions.}
				\label{figRnaGlm}
			}
	\end{figure}

	\subsection{Orientation bias}
	\citet{Shan2011} reported that inducible proviruses were oriented in the same strand as the host cell genes into which they had integrated more often than chance. This orientation bias was still reproduced after our reprocessing of the \Bcl{} sample from  \citet{Shan2011}. However, the proportion of provirus oriented in the same strand as host genes did not differ significantly from 50\% in the other samples (Figure \ref{figOrient}). Perhaps orientation bias and transcriptional interference are especially sensitive to parameters of the model system.
	\begin{figure}
		\centering
			\floatbox[{\capbeside\thisfloatsetup{capbesideposition={right,top}}}]{figure}[\FBwidth]{
				\includegraphics[width=0.5\textwidth]{strandBias.pdf} %REMOVE%
			}{
				\caption[Strand orientation and latency]{The proportion of provirus integrated in the opposite strand compared to cellular genes in silent/inducible (blue) and expressed (red) samples. Error bars show the 95\% Clopper-Pearson binomial confidence interval.}
				\label{figOrient}
			}
	\end{figure}


	\subsection{Gene deserts}
	\citet{Lewinski2005} reported increased latency in gene deserts. In the collected data, integration outside known genes was associated with latency (Fisher's exact test, $p<10^{-6}$).  This seemed to largely be driven by the \Active{} and \Resting{} samples with significant association found individually in only those two samples (both $p<10^{-8}$) and no significant association observed in the other three samples (Figure \ref{figGeneDeserts}A). Looking only at integration sites outside genes, silent sites in the \Resting{} sample had a mean distance to the nearest gene 2.5 times greater than that of expressed sites (95\% CI: 2.2--$6.2\times$, $p<10^{-6}$, Welch two sample t-test on log transformed distance) (Figure \ref{figGeneDeserts}B). The \Active{} sample had a small difference that did not survive Bonferroni correction.

	\citet{Lewinski2005} also reported decreased latency near CpG islands and reasoned this was tied to the increased latency in gene deserts.  In the \Resting{} sample, silent sites were on average further from CpG islands than expressed sites (Bonferroni corrected Welch's two sample T test, $p=0.006$), but there was no significant relationship between silent/inducible status and log distance to CpG island after Bonferroni correction if the integration site's location inside or outside of a gene was accounted for first (analysis of deviance).

	\begin{figure}
		\centering
			\includegraphics[width=\textwidth]{geneDesert.pdf} %REMOVE%
		\caption[Genes and latency]{(A) The proportion of provirus integrated outside genes in silent/inducible (blue) and expressed (red) samples. Error bars show the 95\% Clopper-Pearson binomial confidence interval. (B) The nearest distance to any gene for integration sites (points) outside genes in the five samples. Points are spread in proportion to kernel density estimates. Horizontal lines indicate sample means where there was a significant difference in means between silent/inducible and expressed provirus (black) or no significant difference (grey).}
		\label{figGeneDeserts}
	\end{figure}

	\subsection{Alphoid repeats} %\citep[reviewed in][]{Waye1987}
		Alphoid repeats are repetitive DNA sequences found largely in the heterochromatin of centromeres \citep{Waye1987}. Integration near heterochromatic alphoid repeats has been reported to associate with latency \citep{Jordan2003,Lewinski2005,Pace2012}. Looking only at uniquely mapping sites, there was no statistically significant association between latency and location inside an alphoid repeat in pooled or individual samples (Fisher's exact test).


		Since alphoid repeats are both problematic to assemble in genomes and difficult to map onto, we reasoned that some alphoid hits might be lost or miscounted in the filtering procedures of the standard workup. To counteract this, we treated each sequence read as an independent observation of a proviral integration and included sequence reads with more than one best scoring alignment. For multiply aligned reads, we considered the read to have been inside an alphoid repeat if any of its best scoring alignments fell within a repeat.  We found 74 reads with potential alphoid mappings.  Integration inside alphoid repeats was significantly associated with the expression status of a provirus in the \Resting{}, \Jurkat{} and \Memory{} datasets (Bonferroni corrected Fisher's exact test, all $p < 0.05$) and approached significance in the \Active{} dataset ($p=0.053$) (Figure \ref{figAlphoid}). The \Bcl{} data did not contain any integration sites in alphoid repeats, probably due to 1) the relatively low number of integration sites in the dataset and 2) to the requirement for cleavage at two Pst1 restriction sites, which are not found in the consensus sequence of alphoid repeats \citep{Jurka2005}. Of the 1340 repeat types in the RepeatMasker database \citep{Jurka2005}, only alphoid repeats achieved a significant association with proviral expression in more than two datasets.

	\begin{figure}
		\centering
			\floatbox[{\capbeside\thisfloatsetup{capbesideposition={right,top}}}]{figure}[\FBwidth]{
				\includegraphics[width=0.5\textwidth]{alphoidBar.pdf} %REMOVE%
			}{
				\caption[Alphoid repeats and latency]{The proportion of integration sites with matches in alphoid repeats in silent/inducible (blue) and expressed (red) cells in five samples. Error bars show the 95\% Clopper-Pearson binomial confidence interval. Asterisks indicate significant associations between integrations within an alphoid repeat and proviral expression status (Bonferroni corrected Fisher's exact test $p<0.05$).}
				\label{figAlphoid}
			}
	\end{figure}

	\subsection{Acetylation}
	Histone marks or chromatin remodeling, especially involving the key ``Nuc-1'' histone near the transcription start site in the viral LTR, appear to affect viral expression \citep{Verdin1993,Lint1996,Pearson2008}. Based on this effect, histone deacetylase inhibitors have been developed as potential HIV treatments and show some promise in disrupting latency \citep{Archin2012}. In these genome-wide datasets, we do not have information on the state of individual LTR nucleosomes. However, repressive chromatin does seem to spread to nearby locations if not blocked by insulators \citep{Muller1930,Gaszner2006} and the state of neighboring chromatin could affect proviral transcription independently of provirus-associated histones. 

	We found that the number of ChIP-seq reads near an integration site from several histone acetylation marks (Figure \ref{figCor}) were associated with efficient expression in the \Active{}, \Resting{} and \Memory{} samples. H4K12ac had the strongest association (Bonferroni corrected Fisher's method combination of Spearman's $\rho$, $p<10^{-25}$) with silence/latency (Figure \ref{figAcetylation}A).
	
	Although the appearance of several significantly associated acetylation marks might suggest acetylation exerts a considerable effect on the expression of a provirus, there are strong correlations among these marks, so their effects may not be independent.  To account for the correlations between these variables, we performed a principal component analysis (PCA) to convert the correlated acetylation marks into a series of uncorrelated principal components that capture much of the variance within a few components. Here, the first principal component explained 59\% of the variance and the first ten components 84\%. Several of these principal components again displayed significant associations with latency in the \Active{}, \Resting{} and \Memory{} samples but no significant correlations in the \Bcl{} or \Jurkat{} samples (Figure \ref{figAcetylation}B). A logistic regression of expression status on the first ten principal components and sample did not reduce misclassification error from a base model including only sample in 480-fold cross validation (base model misclassification error: 36.4\%, PCA model: 36.5\%). This suggests that acetylation of neighboring chromatin does not exert strong effects on latency in all samples.

	\begin{figure}
		\centering
			\includegraphics[width=\textwidth]{acetylation.pdf} %REMOVE%
		\caption[Acetylation and latency]{(A) The number of ChIP-seq reads for H4K12ac, the histone mark with the lowest Fisher's method $p$-value for correlation with latency, within 50,000 bases across the five samples. Integration sites (points) are spread in proportion to kernel density estimates. Horizontal lines indicate sample means where there was a significant difference (black) in means between silent/inducible and expressed provirus or no significant difference (grey). (B) The correlation (points) and its 95\% confidence interval (vertical lines) between principal components of acetylation and silent/inducible status for each of the five samples. Red indicates correlations with a Bonferroni-corrected $p$-value $< 0.05$.}
		\label{figAcetylation}
	\end{figure}



	\subsection{Clustering}
	 We reasoned that if there was a strong relationship between latency and chromosomal position, then integration sites that are near one another on the same chromosome should share the same expression status more often than expected by chance. To test this, we compared how often pairs of proviruses shared the same expression status in relation to the distance between the two sites (Figure \ref{figNeighbor}). Pairs of sites with little distance between integration locations did share the same expression status more often than expected by chance (e.g.\  neighbors closer than 100bp, Fisher exact test $p=0.0002$). Breaking out the data to separate between sample and within sample pairings showed that this matching was limited to neighbors within the same experimental model (Figure \ref{figNeighbor}), emphasizing that chromosomal environment does appear to influence latency, but the factors involved differ among experimental models of latency. 

	\begin{figure}
		\centering
			\floatbox[{\capbeside\thisfloatsetup{capbesideposition={right,top}}}]{figure}[\FBwidth]{
				\includegraphics[width=0.5\textwidth]{neighborMatch.pdf} %REMOVE%
			}{
				\caption[Shared expression status between near neighbors]{The ratio of the number of pairs of proviruses with matching expression status to the number of matches expected by random pairings given the frequency of silent/inducible proviruses. All possible pairs of proviruses integrated within a given distance of each other on the same chromosome (red line) were separated into two sets; one with both proviruses from within the same cell culture model and one with proviruses paired between two different cell culture models (black lines).  The shaded region shows the 95\% Clopper-Pearson binomial confidence interval for within and between sample pairings. The dashed horizontal line shows the ratio of 1 expected if there is no association between the expression status of neighboring proviruses.}
			\label{figNeighbor}
			}
	\end{figure}


\section{Conclusions}
	Here we compared the latency status of HIV-1 proviruses in five model systems with the genomic features surrounding their integration sites. Surprisingly, no relationships between genomic features near the integration location and latency achieved significance in all models. Proviruses from the same cellular model integrated in nearby positions did share the same latency status much more often than predicted by chance, indicating the existence of local features influencing latency, but these were not consistent among models. This suggests that whatever features are affecting latency are highly local and model-specific, and that we may not have access to all relevant chromosomal features \citep[e.g.\ ][]{Lassen2006, Dieudonne2009,Siliciano2011,Lusic2013}. %NEW

	%In addition, latency may be affected by other considerations including viral RNA nuclear export \citep{Lassen2006}. \citep{Marco2008} %%NEW%%

	In addition to differences in experimental conditions, methodological issues have the potential to obscure patterns. Examples include multiply infected cells, inactivated viruses and inaccurate assessment of HIV gene activity---each of these are discussed below.

	A latent provirus integrated into the same cell as an expressed provirus will be erroneously sorted as expressed, potentially confounding analysis. A low multiplicity of infection (MOI) will help to avoid this problem, but there is still the potential for a significant proportion of the cells studied to contain multiple integrations. This problem arises because although cells with multiple integrations form a small proportion of total cells, most of the total are cells lacking an integrated provirus and thus are excluded by experimental design. For example, assuming integrations are Poisson distributed with an MOI of 0.1 (1 integration per 10 cells), 90.5\% of cells will not contain a provirus, 9\% of cells will contain one proviral integration and 0.5\% of cells will contain multiple integrations. The cells without an integration are not amplified by HIV-targeted PCR leaving only 9.5\% of the total cells. Of these cells actually under study, 4.9\% will contain multiple integrations. Thus the signal from expressed proviruses may be muted by the presence of latent proviruses in the expressed population. %If 50\% of integrations are latent then 2.5\% of `active' viruses will actually be latent passengers. %If the MOI is 0.5, then 22\% of infected cells will contain more than one integration. [NEED LATENCY RATE TO STICK IN ACTUAL NUMBERS. COULD DO A FEW CALCS IN SUPPLEMENT]] These passengers can contaminate the active virus portion with latent integrations. %1-dpois(1,0.1)/(1-dpois(0,0.1))

	The replication cycle of HIV is error prone, and a significant proportion of virions contain mutated genomes \citep{Mansky1995}. In studies that do not check for inducibility, mutant proviruses integrated in regions of the genome otherwise favorable to proviral expression can be sorted into the latent pool due to mutational inactivation. This problem of inactivated provirus is worse when latent provirus are rare and exacerbated further when looking at latency in the cells of HIV patients due to selective enrichment of inactivated proviruses incapable of spreading infection \citep{Chun1997}. Here, the effects of mutation are minimized in the datasets that required inducible viral expression (\Jurkat{}, \Bcl{}, \Memory{}) but may be a confounder in the two datasets that were sorted based on lack of viral expression only (\Active{}, \Resting{}). 

	Inaccurate staining or leaky markers may also result in misclassification of proviruses. False positives and false negatives will result in incorrectly sorted latent and expressed integrations. For example, if 5\% of cells not containing Gag are labeled as Gag+ and there are an equal amount of latent and expressed integration sites, then 4.8\% of integrations labeled expressed will actually be latent. If a category is rare, false staining has even greater potential to cause error. For example, if only 5\% of sites are latent and a Gag stain has a false negative rate of 5\%, then we would expect 48.7\% of sites classified as latent to actually be mislabeled expressed integrations.

	Attempts to induce latent proviruses in patients have so far focused on using histone deacetylase inhibitors, raising interest in associations with histone acetylation in these data. An important caveat in results from these genome-wide data is that histone modification near the integrated provirus may not be representative of modification within the provirus at the key ``Nuc-1'' nucleosome of the transcription start site \citep{Lint1996}, though local correlations in chromatin states are well established from studies of position effect variegation \citep{Muller1930,Gaszner2006}. We found that some histone acetylation marks were significantly associated with viral expression in some but not all samples (Figures \ref{figCor}, \ref{figAcetylation}). This lack of association may be due to a lack of power in these studies, but the confidence intervals suggest that any correlations between acetylations and latency are unlikely to be strong.  These weak correlations raise the possibility that there are populations of latent proviruses that are not associated with acetylation and may not be inducible by histone deacetylase inhibitors.


	This study highlights that the choice of model system can have a large effect on measurements of latency. Further studies are needed to determine which \textit{in vitro} models best reflect latency \textit{in vivo}. Different cell models may report genuinely different mechanisms of latency. While we did see some relationship between histone acetylation and latency, paralleling a recent clinical trial of SAHA \citep{Archin2012}, associations with histone acetylation did not explain a large fraction of the difference between latent and expresssed proviruses in any of the five models. One possible explanation is that there  may be multiple mechanisms that maintain proviruses in a latent state. To be successful, shock-and-kill treatments must induce and destroy all latent proviruses to eliminate HIV from an infected individual, raising the question of whether multiple simultaneous inducing treatments will be necessary. %NEW
	
	%Genomic features provide a averaged overview in certain cell type and conditions.


\section{Availability of supporting data}
Sequence reads from the \Memory{} sample reported here, the \Resting{} and \Active{} data reported by \citet{Pace2012}, the \Bcl{} data reported by \citet{Shan2011} and  reprocessed data originally reported by \citet{Lewinski2005} are available at the Sequence Read Archive under accession number SRP028573. 

%\section{Author's contributions}
	%SS-M led the computational analysis, with assistance from CCB and NM.  MKL, DL and JG analyzed integration sites using IonTorrent sequencing.  MF, AB and VP prepared DNA from latent and activated T cells using the \Memory{} model.  LS, RFS, MJP, LMA and UO'D contributed data and suggestions.  SS-M, KEO and FDB planned the overall study, and SS-M and FDB wrote the paper. All authors read and approved the final manuscript. 

%%%%%%%%%%%%%%%%%%%%%%%%%%%
\section{Acknowledgements}
  We would like to thank Werner Witke for assistance with IonTorrent sequencing. This work was supported in part by NIH grants R01 AI 052845-11 to FDB, R21AI 096993 and K02AI078766 to UO'D, 5T32HG000046 to SS-M, AI087508 to VP and R01AI038201 to JG, the Penn Genome Frontiers Institute, the University of Pennsylvania Center for AIDS Research (CFAR) P30 AI 045008 and the University of California, San Diego, CFAR P30 AI036214. %Acknowledgements from other labs, UO'D: R21AI 096993 and K02AI078766, VP: AI087508, NIH grant R01AI038201 to JG ASK RICK, technical assistance of Werner Witke for assistance with running the IonTorrent?


\end{document}

