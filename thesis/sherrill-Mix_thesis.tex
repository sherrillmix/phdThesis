%%% PREAMBLE

\documentclass[11pt]{report} % This is in order to have the chapter feature and page numbers in the same location on even and odd pages. 
% Acceptable font sizes are 10 to 12 pts
\usepackage[pdfusetitle,pdfauthor={Scott Sherrill-Mix},hidelinks]{hyperref}  %[[update with title]]
\usepackage{penn_thesis_style}

% Defining variables to be used throughout the document for personalization
\newcommand{\mylowertitle}{Latency, expression and splicing during infection with HIV} 
\def\myauthor{Scott Sherrill-Mix} \def\myauthorfull{Scott A. Sherrill-Mix}
\def\mysupervisorname{Frederic D. Bushman, Ph.D.}
\def\mysupervisortitle{Professor of Microbiology}
\def\gradchairname{Li-San Wang, Ph.D.}
\def\gradchairtitle{Associate Professor of Pathology and Laboratory Medicine}
\def\mydepartment{Genomics and Computational Biology}
\def\myyear{2015}
\def\signatures{46 pt} % Space to accommodate the signatures, you can fiddle with this as you like
\def\mycommittee{Nancy Zhang, Ph.D. Associate Professor of Statistics \\[\baselineskip] %, University of Pennsylvania
	Yoseph Barash, Ph.D., Assistant Professor of Genetics\\[\baselineskip]%, University of Pennsylvania
	Kristen Lynch, Ph.D., Professor of Biochemistry and Biophysics\\[\baselineskip] %, University of Pennsylvania
	Michael Malim, Ph.D., Professor of Infectious Diseases, King's College London
}
\def\mydedication{Dedicated to William Maurer, Gayle Maurer \& Michele Sherrill-Mix}
\def\myacknowledgements{I would like to thank 

Rick

Bushman lab Chris wetlab
collaborators

Friends in the Bushman lab and classmates in GCB.

My committee---Nancy Zhang, Yosephy Barash, Kristen Lynch and Michael Malim---have provided guidance and encouragement. Many faculty of GCB mentoring and teaching. Hannah Chervitz and Tiffany Barlow for managing everything and helping manage the layers of bureacracy. Funding from the HIV Immune Networks Team (HINT) consortium P01 AI090935 and NRSA computational genomics training grant T32 HG000046. 

Ram Myers and Mike James

Xiaofen and Otto


\ldots

}
\def\myabstract{ %%ABSTRACT_START%%
	Over 35 million people are living with human immunodeficiency virus (HIV-1). The mechanisms causing integrated provirus to become latent, the diversity of spliced viral transcripts and the cellular response to infection are not fully characterized and hinder the eradication of HIV-1. We applied high-throughput sequencing to investigate the effects of host chromatin on proviral latency and variation of expression and splicing in both the host and virus during infection.

	To evaluate the link between host chromatin and proviral latency, we compared genomic and epigenetic features to HIV-1 integration site data for latent and active provirus from five cell culture models. Latency was associated with chromosomal position within individual models. However, no shared mechanisms of latency were observed between cell culture models. These differences suggest that cell culture models may not completely reflect latency in patients.

	We carried out two studies to explore mRNA populations during HIV infection. Single-molecule amplification and sequencing revealed that the clinical isolate \hivEight{} produces at least 109 different spliced mRNAs. Viral message populations differed between cell types, between human donors and longitudinally during infection. We then sequenced mRNA from control and \hivEight{}-infected primary human T cells. Over 17 percent of cellular genes showed altered activity associated with infection. These gene expression patterns differed from HIV infection in cell lines but paralleled infections in primary cells. Infection with \hivEight{} increased intron retention in cellular genes and abundance of RNA from human endogenous retroviruses. We also quantified the frequency and location of chimeric HIV-host RNAs. These two studies together provided a detailed accounting of both \hivEight{} and host expression and alternative splicing.

	A more cost-effective method of detecting viral load would aid patients with poor access to healthcare. We developed improved methods for assaying HIV-1 RNA using loop-mediated isothermal amplification based on primers targeting regions of the HIV-1 genome conserved across subtypes. Combined with lab-on-a-chip technology, these techniques allow quantitative measurements of viral load in a point-of-care device targeted to resource-limited settings.

	This work disclosed novel HIV-host interactions and developed techniques and knowledge that will aid in the study and management of HIV-1 infection.

	%Replication of the human immunodeficiency virus requires integration into the host genome, expression and splicing of viral genes and usurpation of host cell machinery, all while avoiding host immune response.

	%HIV replication is a battle between the integrated provirus and innate and adapted immunity of the host.

	%After integrating into host chromosome, the human immunodeficiency virus (HIV-1) commandeers host cell machinery to avoid immune response and 

	%With its small genome size, the human immunodeficiency virus (HIV-1) utilizes alternative splicing 
	%integration, expression, splicing
	%Integration, alternative splicing and usurpation of cellular machinery all while avoiding immune response are integral to the HIV-1 replication cycle.

	%HIV-1 replicatino relies on integration into host chromosome and usurpation of cellular machinery while avoiding immune response.
	%Integration into host chromosomes and 
	%The human immunodeficiency virus (HIV-1) replicates by integrating into host chromosomes and commandeering host cell machinery. It is unclear how the integration location affects latency or how the cell is commandeereed.
	
	%The human immunodeficiency virus (HIV-1) integrates into host chromosomes and then can commandeer host cell machinery to express viral genes and replicate the virus or alternatively can remain latently integrated for years.

	%To replicate, the human immunodeficiency virus (HIV-1) integrates a DNA copy of its RNA genome into host cell chromosomes. This provirus can commandeer host cell machinery to express viral genes and produce virions or alternatively remain latent for years. Here, we applied high-throughput sequencing techniques to investigate the latency, expression and splicing of HIV-1 and its effects on host cells.

	% Here, we applied high-throughput sequencing techniques to investigate the effects of host chromatin on proviral latency and the interplay of gene expression and alternative splicing between host cell and HIV-1. 
	%Over 100 million people are infected with human immunodeficiency virus (HIV-1). The mechanisms causing integrated provirus to become latent, the diversity of spliced viral transcripts and the cellular response to infection are not fully characterized and contribute to the difficulty in eradicating HIV-1. Here, we applied high-throughput sequencing techniques to investigate the effects of host chromatin on proviral latency and how expression and splicing vary in both the host and virus during infection. 

	%We investigated the effect of the chromosomal location of proviral integration on latency. We compared genomic and epigenetic features to integration site data for latent and active provirus from five cell culture models. Latency was associated with chromosomal position within individual models. However, no shared mechanisms of latency were observed between cell culture models. These differences suggest that cell culture models may not reflect latency in patients and that an easily-targetable mechanism for the induction of latent HIV is unlikely.

	%Latent provirus profoundly hinder eradication yet the mechanisms which cause latency are not well understood.
	 %To evaluate the link between host chromatin and proviral latency, we compared genomic and epigenetic features of host chromatin to HIV-1 integration site data for latent and active provirus from five cell culture models. Latency was associated with chromosomal position within individual models. However, no shared mechanisms of latency were observed between cell culture models. These differences suggest that cell culture models may not completely reflect latency in patients.

	%We carried out two studies to explore mRNA populations during HIV infection. Single-molecule amplification and sequencing revealed that the clinical isolate \hivEight{} produces at least 109 different spliced mRNAs. Viral message populations differed between cell types, between human donors and longitudinally during infection. We then sequenced mRNA from control and \hivEight{}-infected primary human T cells. Over 17 percent of cellular genes showed altered activity associated with infection. These gene expression patterns differed from HIV infection in cell lines but paralleled infections in primary cells. Infection with \hivEight{} was associated with increased intron retention in cellular genes and an increased abundance of RNA from human endogenous retroviruses. After filtering sequencing artifacts, we were able to quantify the frequency and location of chimeric HIV-host RNAs. These two studies together provided a detailed accounting of both \hivEight{} and host expression and alternative splicing.

	 %A more cost-effective method of detecting viral load would aid patients with poor access to healthcare. We developed improved methods for assaying HIV RNA using loop-mediated isothermal amplification based on primers targeting regions of the HIV-1 genome conserved across subtypes. Combined with lab-on-a-chip technology, these techniques allow quantitative measurements of viral load in a point-of-care device targeted to resource-limited settings.

	%This work disclosed novel HIV-host interactions and developed techniques and knowledge that will aid in the study and management of HIV-1 infection.

	%These studies use deep sequencing and bioinformatics to aid in the understanding and management of HIV replication.
%These studies illustrate the power of deep sequencing and bioinformatics in understanding and managing HIV replication and developing approaches to management.

%No more than 350 words. It is normally a single paragraph, consists of four parts: the statement of the problem; the procedure and methods used to investigate the problem; the results of the investigation; and the conclusions. The abstract is published online by ProQuest in ``Dissertation Abstracts International'', providing information to interested readers about the general content of the dissertation.
} %%ABSTRACT_END%%




%abbreviations
\newcommand{\cdFour}{CD4$^+$}
\newcommand{\hivEight}{HIV$_{89.6}$}
\newcommand{\hivLai}{HIV$_{\textrm{LAI}}$}
\newcommand{\threePrime}{3\ensuremath{'}}
\newcommand{\fivePrime}{5\ensuremath{'}}
\newcommand{\degree}{\ensuremath{^\circ}}
\newcommand{\uM}{\ensuremath{\mu}M}
\newcommand{\uL}{\ensuremath{\mu}L}
\newcommand{\ug}{\ensuremath{\mu}g}
\newcommand{\approximately}{\mathord{\sim}}
\newcommand{\hivNL}{HIV$_{\text{NL4-3}}$}
\newcommand{\gag}{\emph{gag}}
\newcommand{\pol}{\emph{pol}}
\newcommand{\tat}{\emph{tat}}
\newcommand{\rev}{\emph{rev}}
\newcommand{\nef}{\emph{nef}}
\newcommand{\vpr}{\emph{vpr}}
\newcommand{\vif}{\emph{vif}}
\newcommand{\vpu}{\emph{vpu}}
\newcommand{\env}{\emph{env}}


%%% BEGIN DOCUMENT

\begin{document}
% Making tables and figures numbered continuously
%\makeatletter
%\@removefromreset{table}{chapter}
%\makeatother
%\renewcommand{\thetable}{\arabic{table}}
%\makeatletter
%\@removefromreset{figure}{chapter}
%\makeatother
%\renewcommand{\thefigure}{\arabic{figure}}

%do some calculation based on user inputs
\newcommand{\mytitle}{\expandafter\MakeUppercase\expandafter{\mylowertitle}} % Make sure this is in all caps 
\newlength{\superlen}   % a "scratch" length
\settowidth{\superlen}{\mysupervisorname, \mysupervisortitle} % Width of signature line for supervisor
\newlength{\chairlen}   % a "scratch" length
\settowidth{\chairlen}{\gradchairname, \gradchairtitle} % Width of signature line for supervisor
\newlength{\maxlen}
\setlength{\maxlen}{\maxof{\superlen}{\chairlen}}

%set pdf title and author
\hypersetup{pdftitle={\mylowertitle},pdfauthor={\myauthor}}

%% PRELIMINARY PAGES

\pagenumbering{roman}
\pagestyle{plain}

% TITLE PAGE
\begin{titlepage}
	\thispagestyle{empty} % No page numbers on title page, as per Manual page 8
	\begin{center}

	\onehalfspacing

	\mytitle

	\myauthor

	A DISSERTATION

	in 

	\mydepartment 

	%For the Graduate Group in Managerial Science and Applied Economics 

	Presented to the Faculties of the University of Pennsylvania

	in 

	Partial Fulfillment of the Requirements for the

	Degree of Doctor of Philosophy

	\myyear

	\end{center}

	\vfill % Here to make sure the page is filled

	\begin{flushleft}

	Supervisor of Dissertation\\[\signatures] % Space for signature, you can fiddle with this as you like

	\renewcommand{\tabcolsep}{0 pt}
	\begin{table}[h]
	\begin{tabularx}{\maxlen}{l}
	\toprule
	\mysupervisorname, \mysupervisortitle\\ %Space between advisor and graduate chair, you can fiddle with this as you like
	\end{tabularx}
	\end{table}

	Graduate Group Chairperson\\[\signatures] % Space for signature, you can fiddle with this as you likee

	\begin{table}[h]
	\begin{tabularx}{\maxlen}{l}
	\toprule
	\gradchairname, \gradchairtitle\\ %Space between advisor and graduate chair, you can fiddle with this as you like
	\end{tabularx}
	\end{table}
	\singlespacing

	Dissertation Committee: % No signature necessary

	\mycommittee

	\end{flushleft}
\end{titlepage}

%title page should be counted
\addtocounter{page}{1}

% COPYRIGHT NOTICE (optional)

\doublespacing

\thispagestyle{empty} % No page number as per Manual, p. 11

\vspace*{\fill}

%copyright should be page ii
\begin{flushleft}
	\mytitle

	 \copyright \space COPYRIGHT
	 
	\myyear

	\myauthorfull\\[24 pt] % If traditional copyright then delete everything below here, but keep \end{flushleft}

	This work is licensed under the \\
	Creative Commons Attribution \\
	NonCommercial-ShareAlike 3.0 \\
	License

	To view a copy of this license, visit

	\url{http://creativecommons.org/licenses/by-nc-sa/3.0/}
\end{flushleft}
\pagebreak 

% Changing formatting for preliminary pages (NOT OPTIONAL)
\newenvironment{preliminary}{}{}
\titleformat{\chapter}[hang]{\large\center}{\thechapter}{0 pt}{}
\titlespacing*{\chapter}{0pt}{-33 pt}{6 pt} % The key value here is the -33 pts, I got to it by old fashioned measuring with a ruler....
\begin{preliminary}

% DEDICATION (OPTIONAL)

\begin{center}
\vspace*{\fill}
\textit{\mydedication}
\vspace*{\fill}
\end{center}

% ACKNOWLEDGEMENT (OPTIONAL)

\clearpage
\chapter*{ACKNOWLEDGEMENTS}
%\addcontentsline{toc}{chapter}{ACKNOWLEDGEMENT} % This is to include this section in the Table of Contents

\myacknowledgements


% ABSTRACT

\clearpage
\chapter*{ABSTRACT}
\addcontentsline{toc}{chapter}{ABSTRACT} % This is to include this section in the Table of Contents
\begin{center}
\mytitle

\myauthor

\mysupervisorname

\end{center}

\myabstract

% TABLE OF CONTENTS

\clearpage
\microtypesetup{protrusion=false} % disables protrusion locally in the document
\singlespacing


\tableofcontents

% LIST OF TABLES

\clearpage
\listoftables
\addcontentsline{toc}{chapter}{LIST OF TABLES}

% LIST OF ILUSTRATIONS

\clearpage
\listoffigures
\addcontentsline{toc}{chapter}{LIST OF ILLUSTRATIONS}
\microtypesetup{protrusion=true} % enables protrusion
\doublespacing

% PREFACE (OPTIONAL)

%\clearpage
%\chapter*{PREFACE (optional)}
%\addcontentsline{toc}{chapter}{PREFACE} % This is to include this section in the Table of Contents

\end{preliminary}

%% MAIN TEXT
\newenvironment{mainf}{}{}
\titleformat{\chapter}[hang]{\large\center}{CHAPTER \thechapter}{0 pt}{ : }
\titlespacing*{\chapter}{0pt}{-29 pt}{6 pt} % The key value here is the -29 pts, I got to it by old fashioned measuring with a ruler....
\begin{mainf}

\newpage
\pagenumbering{arabic}
\pagestyle{plain} % This has to be repeated here because the lists change the style

% Dutch style of paragraph formatting, i.e. no indents. 
\setlength{\parskip}{10 pt} % Same as Word file
\setlength{\parindent}{0pt}




\subfile{intro/intro}

\subfile{rnaSeq/rnaSeq}

\subfile{latency/latency}

\subfile{lamp/lamp}

\subfile{pacBio/pacBio}

\subfile{conclusion/conclusion}

\end{mainf}

%% APPENDICES (OPTIONAL)
\appendix

\begin{appendixf}

%\addtocontents{toc}{\protect\setcounter{tocdepth}{-1}} % This is to fix how appendices are shown in ToC
%\chapter{Appendix}
%\addtocontents{toc}{\protect\setcounter{tocdepth}{1}} % This is to bring things back to normal
%\addcontentsline{toc}{chapter}{APPENDIX} % You have to write here everything you want the ToC to show including \thechapter if you want numbering
%\addtocontents{toc}{\protect\setcounter{tocdepth}{-1}} % This is so sections in the appendix are not shown in the ToC


\subfile{appendices/latencyReport/AdditionalFile4.tex}

%\addtocontents{toc}{\protect\setcounter{tocdepth}{1}} % This is to bring things back to normal
\end{appendixf}

%% BIBLIOGRAPHY


\begin{bibliof}

\bibliographystyle{unsrtnat}
\bibliography{refs} % This is the filename of the bibtex bibliography file (it has to be in the same directory as the main LaTeX file)

\end{bibliof}
%% INDEX (OPTIONAL) - I do not really know how to code this, sorry

\end{document}
