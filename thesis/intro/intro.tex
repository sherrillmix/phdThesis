\documentclass[../sherrill-Mix_thesis.tex]{subfiles}
\begin{document}
\graphicspath{{im/}{intro/im/}}
\chapter{Introduction}
\section{Impact of HIV}

\section{The HIV virus}

\section{Integration and latency}


\section{HIV splicing}

\section{Host cell interactions}
%HIV may suppress miRNA silencing pathway \citep{Bennasser2005,Triboulet2007,Qian2009} but controversial \citep{Lin2007}
%TAR-dervied miRNA might change gene expression particularly apoptosis pathway \citep{Klase2009}.

%Interferon inducible genes
%change splicing factors \citep{Zhu2002,Dowling2008}

%repetitive elements and vaccination

\section{HIV detection}
%http://www.medscape.com/viewarticle/715166_2
	Immunoassays provide cheap immediate testing of HIV infection in patients. These tests are based on the enzyme-linked immunosorbent assay (ELISA), using an enzyme linked to an antibody to produce a detectable signal in the presence of antigen \citep{Yalow1960,Engvall1971,VanWeemen1971}. %If the labeled antibodies are targeted to human antibodies then a patient's immune response to an antigen can be measured.
	
	The isolation of HIV \citep{Barre-Sinoussi1983,Gallo1983,Popovic1984,Levy1984} allowed the production of large quantities of virions. These virions were bound to a substrate, sera from patients added and any patients antibodies sensitive to HIV allowed to bind. Any unbound antibodies were washed away. Then a peroxidase enzyme-labeled antibody targeted to human antibody bound was added, allowed to bind and the unbound antibodies again washed away. Any HIV-targeted patient antibodies would bind the antigen and be bound in turn by the peroxidase-labeled antibody and the peroxidase would then change the color of media \citep{Safai1984,Sarngadharan1984}. These tests had a large false positive rate and the standard procedure was to perform multiple ELISA tests follow by a Western blot test \citep{Towbin1979,CDC1985} but false positives were still prevalent \citep{Burke1986}. More conservative criteria and cleaner lab procedures reduced false positives \citep{Burke1988}. These assays have been developed to fourth generation \citep{Chappel2009} with more sensitive and specific detection of patient antibodies and earlier detection using antibodies against the HIV capsid protein \citep{Weber1998,Weber2002}.  

	Slightly less specific but rapid immunoassays providing results in 30 minutes have been developed to allow point-of-care testing with many fewer patients lost to follow up prior to delivery of results \citep{Kassler1995,CDCP1998,CDCP2002}. Rapid tests detecting HIV in oral fluids have been developed and obviate the need for a blood draw \citep{Gallo1997,Delaney2006,SemaBaltazar2014}. These rapid care tests allow self testing at home \citep{Granade2004,PantPai2013}.

	Reverse transcription and PCR amplification offers another alternative \citep{Hart1988,Ou1988} but is not currently cost effective for primary patient screening \citep{Long2011}.

	Tests allowing point-of-care qualitative HIV detection are now widespread but point-of-care assays for viral load in a patient exist. In addition, existing laboratory-based tests are relatively expensive and require specialized equipment making access difficult in resource-limited settings \citep{Fiscus2006,Wang2010a}. Without viral load measures, \cdFour{} T cell counts or clinical presentation are used to infer the emergence of drug. These criteria are not specific or sensitive enough without viral load measures so many patients are unnecessarily switched to second line therapy \citep{Mee2008,VanOosterhout2009} or switched too late leading to accumulations of drug resistant mutations \citep{Hosseinipour2009}.  In Chapter \ref{chapLamp}, we design loop-mediated isothermal amplification methods that can be used with microfluidics to create a point-of-care assay of infection and viral load in resource-limited settings.

\section{Contribution summaries}
	Much of this work was performed as part of a large collaboration. It would not tell a complete story in isolation. Therefore, I have preserved the chapters in published form in and detailed my contribution to each project at the start of the chapter.


\end{document}
